\documentclass[12pt]{sciposter}
\usepackage{lipsum}
\usepackage{epsfig}
\usepackage{amsmath}
\usepackage{amssymb}
\usepackage{multicol}
\usepackage{graphicx,url}
\usepackage[utf8]{inputenc}
\usepackage{listings}
\usepackage[brazil]{babel}   
%\usepackage{fancybullets}
\newtheorem{Def}{Definition}
\lstset{ %
basicstyle=\footnotesize,       % the size of the fonts that are used for the code
numbers=left,                   % where to put the line-numbers
extendedchars=true,
numberstyle=\footnotesize,      % the size of the fonts that are used for the line-numbers
stepnumber=1,                   % the step between two line-numbers. If it is 1 each line will be numbered
numbersep=5pt,                  % how far the line-numbers are from the code
showspaces=false,               % show spaces adding particular underscores
showstringspaces=false,         % underline spaces within strings
showtabs=false,                 % show tabs within strings adding particular underscores
%frame=single,           % adds a frame around the code
tabsize=2,          % sets default tabsize to 2 spaces
captionpos=b,           % sets the caption-position to bottom
breaklines=true,        % sets automatic line breaking
breakatwhitespace=false,    % sets if automatic breaks should only happen at whitespace
escapeinside={\%*}{*)},          % if you want to add a comment within your code
literate={á}{{\'a}}1 {ã}{{\~a}}1 {é}{{\'e}}1 {â}{{\^a}}1 {õ}{{\~o}}1 {ç}{{\c{c}}}1 {à}{{\`a}}1 {ú}{{\'u}}1
}


\title{Dragon Souls of Senac: Touhou Edition}
%Título do projeto

\author{Gabriel Nopper Silva, Danilo Makoto Ikuta}
%nome dos autores

\institute 
{Bacharelado em Ciência da Computação\\
Centro Universitário SENAC - Campus Santo Amaro
  (SENAC-SP)\\
  Av. Engenheiro Eusébio Stevaux, 823 -- Santo Amaro, São Paulo -- CEP 04696-000 -- SP -- Brasil}
%Nome e endereço da Instituição

\email{{(ganopper@gmail.com, danilo0195@hotmail.com})}
% Onde você coloca os emails dos integrantes


%\date is unused by the current \maketitle

\rightlogo[1]{respect_cube}
\leftlogo[1]{Senac-logo}
% Exibe os logos (direita e esquerda) 
% Procure usar arquivos png ou jpg, e de preferencia mantenha na mesma pasta do .tex
%%%%%%%%%%%%%%%%%%%%%%%%%%%%%%%%%%%%%%%%%%%%%%%%%%%%%%%%%%%%%%%%%%%%%%%%%%%%%%%%
%%% Begin of Document



\begin{document}
%define conference poster is presented at (appears as footer)

\conference{{\bf PI3 - 2014}, Ciencia da Computação - Senac, 13 de Junho de 2014, São Paulo, Brasil}

%\LEFTSIDEfootlogo  
% Uncomment to put footer logo on left side, and 
% conference name on right side of footer

% Some examples of caption control (remove % to check result)

%\renewcommand{\algorithmname}{Algoritme} % for Dutch

%\renewcommand{\mastercapstartstyle}[1]{\textit{\textbf{#1}}}
%\renewcommand{\algcapstartstyle}[1]{\textsc{\textbf{#1}}}
%\renewcommand{\algcapbodystyle}{\bfseries}
%\renewcommand{\thealgorithm}{\Roman{algorithm}}

\maketitle

%%% Begin of Multicols-Enviroment
\begin{multicols}{3}

%%% Abstract
%%% Introduction
\section{Introducão}

\PARstart{E}{ste} terceiro Projeto Interativo do Bacharelado de Ciência da Computação traz a ideia da visão computacional.
Neste relatório explica-se como com apenas linguagem C, auxiliada das bibliotecas OpenCV e Allegro5, foi possivel criar um jogo de luta com escudo e espada, utilizando como unico controle usuário-maquina uma camera, e dois emissores de luz, uma vermelha e outra verde.

\begin{figure}[!htb]
    \centering
    \includegraphics[scale = 0.5]{Game.png}
    %%talvez precise mudar a escala
    %\label{}
\end{figure}

\section{Objetivos}
O objetivo do projeto é desenvolver um jogo em Linguagem C, utilizando apenas duas bibliotecas externas - OpenCV, referente apenas à captura de imagem (a segmentação das imagens da câmera ficou à cargo dos alunos), e Allegro5, referente à imagem.
O jogo aqui descrito é um jogo de ação, cujo jogador luta contra vários monstros, atacando suas imagens com uma espada(Que trata-se de uma projeção de luz vermelha), por diversas "Fases", cada uma com monstros diferentes que lançam ataques na tela que podem ser bloqueados por um escudo (Que trata-se da projeção de luz verde).
Tem-se como principais objetivos, criar uma detecção rapida e fiel ao movimento do corte com luz vermelha, e da defesa com luz verde, e como objetivos secundários, estruturar um jogo estável, divertido, e com um fluxo rápido.
Para poder identificar essas luzes, o primeiro passo foi converter o espaço de cor de RGB(vermelho, verde, azul) padrão para HSV(matiz, saturação, valor), visto que os testes para identificar apenas a cor verde ou vermelha em RGB apresentavam resultados consideravelmente diferentes quando se trata de pequenas variações na iluminação, que pode ser mais facilmente ajustado no espaço HSV. Durante o processamento de câmera, este se tornou o espaço padrão do projeto. Outro ponto foi utilizar fontes de luz das cores vermelha e verde ao invés de simplesmente um objeto da cor para evitar que a camiseta de um jogador, por exemplo, seja acidentalmente detectada pelo programa.
Além disso, o jogo exibe a silhueta do jogador(em um tom azulado), fazendo uso do método de subração de fundo. No momento em que o programa é inicializado, ele passa um certo período (sem que o jogador apareca na câmera) detectando diversas vezes o fundo(background) e a cada captura é feita uma média dos valores para cada pixel, afim de produzir uma imagem referencial mais estável, que durante o ciclo de câmera, cada pixel é comparado com seu correspondente no quadro(frame) mais atual da câmera. Dependendo da variação nos valores em HSV, um certo pixel pode ser considerado como de fundo(background) ou pertencente ao jogador.\\
O algoritmo de obtenção de fundo usado foi:

\begin {lstlisting}
Obtenção de fundo{
    //Obtenção do HSV
    Dado certo número de vezes{
        Para cada pixel{
            Converte os valores RGB para HSV
            
            Soma os valores à imagem de saida
        }
    }
    
    Para cada pixel{
        Divide o valor pela quantidade de medições
    }
}

Ciclo de Câmera{
    Para cada pixel da imagem de entrada{
        Converte para HSV o pixel
        
        Compara os valores HSV com os parâmetros para a espada
        
        Se(For um pixel da espada){
            Soma as coordenadas (x, y) à respectiva variável
            Incrementa o contador de pontos da espada
        }
        
        Compara os valores HSV com os parâmetros para personagem
        
        Se(For um pixel de personagem){
            Marca como ponto de personagem em uma matriz auxiliar
        }
        
        Senão{
            Marca como ponto de fundo na matriz auxiliar
        }
        
        Identificação de escudo
        //Realiza as mesmas operações da identificação da espada, mas com parâmetros e variáveis diferentes
    }
    
    Se(Número de pontos de escudo for maior que um certo número){
        Calcula as coordenadas(x, y) do centro de massa dos pontos do escudo
    }
    
    Senão{
        Atribui valores inválidos às coordenadas do centro de massa, indicando que não há escudo
    }
    
    Se(Número de pontos de espada for maior que dado número){
        Calcula coordenadas do centro de massa dos pontos da espada
        Insere essas coordenadas numa fila
    }
    
    Senão{
        Insere as coordenadas do ponto mais recente da fila, se houver.
    }
    
    Se(Número de pontos na fila for maior que um certo valor){
        Retira um ponto da fila.
    }
    
    Percorre a fila, desenhando linhas entre os pontos consecutivos, representando o "rastro da espada".
}
\end{lstlisting}

\section{Metodologia}

O jogo propriamente dito roda através de um ciclo principal que se encerra apenas no caso do jogador perder ou ganhar o jogo. Dentro deste ciclo estão diversos outros pequenos ciclos, que fazem o programa executar. Em forma de pseudocódigo, a estrutura geral do ciclo principal ficaria da seguinte forma:


\begin{lstlisting}
Enquanto(Ultimo chefe não derrotado){
    
    // Ciclo dos monstros
    Para (cada monstro na lista){
        Realiza Movimento;
        Avança contador de timer p/ ataque;
        Se (contador de ataque pronto){
            Gera ataque;
            Reseta contador de ataque;
        }
        Desenha bitmap do monstro;
    }
    
    Verifica e limpa monstros mortos;
    
    // Ciclo do chefe
    Se (Há um chefe na lista){
        Se (Vida abaixo de 0){
            Verifica e limpa monstros;
        }
        Se não{
            Executa movimento;
            Avança cont. de timer p/ ataque;
            Se (Contador de ataque pronto){
                Gera ataque;
                Reseta contador de ataque;
            }
        Desenha bitmap do chefe;
        }
    }
    
    //Ciclo de câmera
    Chama captura da câmera;
    
    //Ciclo de ataques
    Para(Cada ataque na lista){
        Move ataque;
        Se(Pos. do Ataque for a Pos. alvo){
            Se(acertou escudo){
                Executa audio;
                Deleta ataque;
            }
            Se não{
                Subtrai dano da defesa;
                Causa Dano;
            }
        }
    }
    
    //Ciclo de interface
    
    Se(Há animações para fazer)
        Executa animações;
        
    Desenha escudo;
    Desenha Elementos de interface;
    
    Joga elementos desenhados na tela;
    
    //Ciclo do jogador
    Para (Cada monstro na lista){
        Verifica se foi acertado;
        //Dentro desta verificação ocorre dano
    }
    
    //ciclo de fim de jogo
    Se(Vida do jogador inferior a 1){
        Encerra jogo por função de morte;
    }
    
}

\end{lstlisting}


\begin{figure}[!htb]
    \centering
    \includegraphics[scale = 0.5]{Death.png}
    %%talvez precise mudar a escala
    %\label{}
\end{figure}

\section{Resultados e Discussão}

O resultado foi bastante próximo do desejado. Embora haja um atraso na detecção natural de qualquer WebCam, o escudo é possivel jogar o jogo mesmo este sendo rapido.
A detecção do escudo ficou extremamente precisa, com uma detecção rapida que permite movimentações rapidas. A detecção da espada, similarmente possui uma detecção rapida e precisa, embora devido a nosso hardware, o usuario deva tomar mais  cuidado com apontar o led frontalmente.
Inicialmente, foi escolhido este projeto, não apenas pela detecção de objetos diversos, mas pela construção de um jogo elaborado e bem construido, algo que foi atingido.
O unico fator não atingido nos resultados esperados foi o hitbox do jogador, gerado pela subtração de fundo. A subtração de fundo do corpo do jogador deveria ser a unica região onde ele fosse acertado, permitindo esquivas no jogo.
Entretanto, subtração de fundo não é um método perfeito, deixando vãos e "Buracos" no hitbox do jogador, especialmente em regiões onde a cor do jogador e do fundo são bem similares.
Mesmo assim, manteve-se a subtração de fundo, não como hitbox, mas apenas como representação grafica do jogador no jogo.

\section{Conclusão}
Chegamos onde queriamos. Isto pode ser dito com certeza.
O conhecimento prévio de allegro foi relevante, entretanto nós dois trabalhamos no BEPiD, e lá realizamos varios pequenos projetos de lógica similar, que nos permitiu realizar este projeto maior facilidade. A mesma lógica usada gerou os ciclos de ataque e monstros.
O grande desafio foi o reconhecimento da Camera, e concluimos que a camera pode ser um instrumento extremamente eficiente, apesar de custoso para a memória. Foi interessante também aprender a composição visual do computador e da WebCam, lições que definitivamente serão uteis no futuro.

%%% References

%% Note: use of BibTeX als works!!

\bibliographystyle{plain}
\begin{thebibliography}{1}

\bibitem{1}
Touhou 6 - Enbodiment of Scarlet Devil
\newblock
\bibitem{2}
Dark Souls, Atlus
\newblock
\bibitem{3}
Multiplos, J.R.R.Tolkien
\newblock
\bibitem{4}
Heredian, PI6 de 2013 do BCC-Senac
\newblock https://github.com/celsovlpss/BCC-2s13-PI6-Heredian



\end{thebibliography}

\end{multicols}

\end{document}

